The {\ttfamily \hyperlink{class_temperature}{Temperature}} class manages the analog readings from an {\itshape L\-M35} temperature sensor device to control the cooler fan speed.

\par
The base calculation for the temperature after the sensor reading is managed by the {\ttfamily Calc\-Temp} method of the \hyperlink{class_temperature}{Temperature} class; the 10-\/bit A\-D conversion generates 1024 possible values (between 0 and 1023) so the base Celsius float conversion formula is\-:\par
 \begin{center}{\bfseries \mbox{[}sensor\-Value / 1024\mbox{]} $\ast$ 5) $\ast$ 100 }\end{center} 

From this first conversion calculation are derived all the other units values\-: Fahrenheit, Kelvin and Rankine following the formulas below\-:\par
 \begin{center}{\bfseries  Fahrenheit = (Celsius $\ast$ 9) / 5) + 32\par
 Kelvin = Celsius -\/ A\-B\-S\-O\-L\-U\-T\-E\-\_\-\-Z\-E\-R\-O\-\_\-\-C\-E\-L\-S\-I\-U\-S\par
 Rankine = (Celsius -\/ A\-B\-S\-O\-L\-U\-T\-E\-\_\-\-Z\-E\-R\-O\-\_\-\-C\-E\-L\-S\-I\-U\-S) $\ast$ 9) / 5\par
 }\end{center}  